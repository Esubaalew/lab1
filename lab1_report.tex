\documentclass[12pt,a4paper]{article}
\usepackage{graphicx}
\usepackage{geometry}
\geometry{margin=1in}
\usepackage{hyperref}
\usepackage{amsmath}
\usepackage{float}

\title{Report for Lab 1\\
Computer Vision Course\\
MSc in Artificial Intelligence}
\author{Esubalew Chekol\\
GSR/6451/17}
\date{August 28, 2025}

\begin{document}

\maketitle

\section*{Introduction}
This report summarizes the implementation and results of basic image processing tasks using MATLAB. The tasks are divided into two main sections: grayscale image sampling and quantization (Task 1), and color image processing (Task 2). The images used are \texttt{trees.jpeg} and \texttt{peppers.jpeg}.

\section*{Task 1: Grayscale Image Sampling and Quantization}

\subsection*{1. Displaying the Original Image}
\textbf{Script:} \texttt{task1\_display.m}

Reads and displays the original grayscale image (\texttt{trees.jpeg}).

% Code
\begin{verbatim}
img = imread('images/trees.jpeg');
imshow(img);
title('Original Image');
\end{verbatim}

% Image
\begin{figure}[H]
    \centering
    \includegraphics[width=0.4\textwidth]{images/out/original_trees.jpeg}
    \caption{Original Grayscale Image}
\end{figure}

\subsection*{2. Image Sampling}
\textbf{Script:} \texttt{task1\_sampling.m}, \texttt{task1\_downsampling.m}

Samples the image by factors of 2 and 4, reducing its resolution.

% Code
\begin{verbatim}
% Factor 2
sampled_img = img(1:2:end, 1:2:end);
imshow(sampled_img);
title('Sampled Image - Factor 2');

% Factor 4
sampled_img_4 = img(1:4:end, 1:4:end);
imshow(sampled_img_4);
title('Sampled Image - Factor 4');
\end{verbatim}

% Images (replace placeholders with actual output)
\begin{figure}[H]
    \centering
    \includegraphics[width=0.4\textwidth]{images/out/sampled_trees.jpeg}
    \caption{Sampled Image - Factor 2}
\end{figure}
\begin{figure}[H]
    \centering
    \includegraphics[width=0.4\textwidth]{images/out/sampled4_trees.jpeg}
    \caption{Sampled Image - Factor 4}
\end{figure}

\subsection*{3. Image Quantization}
\textbf{Script:} \texttt{task1\_quantization.m}

Reduces the number of gray levels to 4.

% Code
\begin{verbatim}
quant_img = uint8(floor(double(img)/64) * 128);
imshow(quant_img);
title('Quantized Image - 4 Gray Levels');
\end{verbatim}

% Image (quantization)
\begin{figure}[H]
    \centering
    \includegraphics[width=0.4\textwidth]{images/out/quantized_trees.jpeg}
    \caption{Quantized Image - 4 Gray Levels}
\end{figure}

\subsection*{4. Binary Conversion}
\textbf{Script:} \texttt{task1\_gray\_2.m}

Converts the image to binary using a threshold of 128.

% Code
\begin{verbatim}
binary_img = uint8(img > 128) * 255;
imshow(binary_img);
title('Binary Image (Threshold at 128)');
\end{verbatim}

% Image (binary)
\begin{figure}[H]
    \centering
    \includegraphics[width=0.4\textwidth]{images/out/binary_trees.jpeg}
    \caption{Binary Image (Threshold at 128)}
\end{figure}

\subsection*{5. Cascaded Sampling and Quantization}
\textbf{Script:} \texttt{task1\_cascade.m}

Applies sampling and quantization sequentially.

% Code
\begin{verbatim}
cascaded_img = img(1:2:end, 1:2:end);
cascaded_img = uint8(floor(double(cascaded_img)/64) * 64);
imshow(cascaded_img);
title('Sampled and Quantized Image');
\end{verbatim}

% Image (cascade)
\begin{figure}[H]
    \centering
    \includegraphics[width=0.4\textwidth]{images/out/cascaded_trees.jpeg}
    \caption{Sampled and Quantized Image}
\end{figure}

\section*{Task 2: Color Image Processing}

\subsection*{1. Displaying the Original Color Image}
\textbf{Script:} \texttt{task2\_original.m}

Reads and displays the original color image (\texttt{peppers.jpeg}).

% Code
\begin{verbatim}
color_img = imread('images/peppers.jpeg');
imshow(color_img);
title('Original Color Image');
\end{verbatim}

% Image (color original)
\begin{figure}[H]
    \centering
    \includegraphics[width=0.4\textwidth]{images/out/original_peppers.jpeg}
    \caption{Original Color Image}
\end{figure}

\subsection*{2. Brightness Adjustment}
\textbf{Script:} \texttt{task2\_brightness.m}

Decreases the brightness of the image.

% Code
\begin{verbatim}
bright_img = color_img - 50;
imshow(bright_img);
title('Brightness Decreased');
\end{verbatim}

% Image (brightness)
\begin{figure}[H]
    \centering
    \includegraphics[width=0.4\textwidth]{images/out/brightness_peppers.jpeg}
    \caption{Brightness Decreased}
\end{figure}

\subsection*{3. Contrast Stretching}
\textbf{Script:} \texttt{task2\_contrast.m}

Stretches the contrast of the image using \texttt{imadjust}.

% Code
\begin{verbatim}
double_img = im2double(color_img);
contrast_img = imadjust(double_img, stretchlim(double_img), []);
imshow(contrast_img);
title('Contrast Stretched Image');
\end{verbatim}

% Image (contrast)
\begin{figure}[H]
    \centering
    \includegraphics[width=0.4\textwidth]{images/out/contrast_peppers.jpeg}
    \caption{Contrast Stretched Image}
\end{figure}

\subsection*{4. Negative Image}
\textbf{Script:} \texttt{task2\_negative.m}

Computes the negative of the image.

% Code
\begin{verbatim}
negative_img = 255 - color_img;
imshow(negative_img);
title('Negative Image');
\end{verbatim}

% Image (negative)
\begin{figure}[H]
    \centering
    \includegraphics[width=0.4\textwidth]{images/out/negative_peppers.jpeg}
    \caption{Negative Image}
\end{figure}

\subsection*{5. RGB Channel Separation}
\textbf{Script:} \texttt{task2\_rgb.m}

Separates and displays the Red, Green, and Blue channels.

% Code
\begin{verbatim}
R = color_img(:,:,1);
G = color_img(:,:,2);
B = color_img(:,:,3);

subplot(1,3,1); imshow(R); title('Red Channel');
subplot(1,3,2); imshow(G); title('Green Channel');
subplot(1,3,3); imshow(B); title('Blue Channel');
\end{verbatim}

% Images (RGB channels)
\begin{figure}[H]
    \centering
    \includegraphics[width=0.3\textwidth]{images/out/red_peppers.jpeg}
    \caption{Red Channel}
\end{figure}
\begin{figure}[H]
    \centering
    \includegraphics[width=0.3\textwidth]{images/out/green_peppers.jpeg}
    \caption{Green Channel}
\end{figure}
\begin{figure}[H]
    \centering
    \includegraphics[width=0.3\textwidth]{images/out/blue_peppers.jpeg}
    \caption{Blue Channel}
\end{figure}

\section*{Conclusion}
This lab provided hands-on experience with fundamental image processing techniques in MATLAB. The tasks covered essential operations such as sampling, quantization, brightness/contrast adjustment, and color channel manipulation, forming a strong foundation for further studies in computer vision.

\end{document}
